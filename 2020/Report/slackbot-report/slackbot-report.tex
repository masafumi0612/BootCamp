\documentclass[12pt]{jsarticle}
\usepackage[dvipdfmx]{graphicx}
\textheight = 25truecm
\textwidth = 18truecm
\topmargin = -1.5truecm
\oddsidemargin = -1truecm
\evensidemargin = -1truecm
\marginparwidth = -1truecm

\def\theenumii{\Alph{enumii}}
\def\theenumiii{\alph{enumiii}}
\def\labelenumi{(\theenumi)}
\def\labelenumiii{(\theenumiii)}
\def\theenumiv{\roman{enumiv}}
\def\labelenumiv{(\theenumiv)}
\usepackage{comment}
\usepackage{url}

%%%%%%%%%%%%%%%%%%%%%%%%%%%%%%%%%%%%%%%%%%%%%%%%%%%%%%%%%%%%%%%%
% sty/ にある研究室独自のスタイルファイル
\usepackage{jtygm}  % フォントに関する余計な警告を消す
\usepackage{nutils} % insertfigure, figref, tabref マクロ

\def\figdir{./figs} % 図のディレクトリ
\def\figext{pdf}    % 図のファイルの拡張子



\begin{document}
%%%%%%%%%%%%%%%%%%%%%%%%%%%%
%% 表題
%%%%%%%%%%%%%%%%%%%%%%%%%%%%
\begin{center}
{\LARGE SlackBotプログラムの報告書}
\end{center}

\begin{flushright}
  2020/7/21\\
  野村 優文
\end{flushright}
%%%%%%%%%%%%%%%%%%%%%%%%%%%%
%% 概要
%%%%%%%%%%%%%%%%%%%%%%%%%%%%

\section{概要}
\label{sec:introduction}
本資料は,2020年度B4新人研修課題の1つで作成した,SlackBotプログラムの報告書である.
本プログラムはチャットツールであるSlack\cite{Slack}を用いる.
また,SlackBotは,ユーザがSlack上で投稿した特定の文章をきっかけとして,Slack上で自動的に返信する機能をもつ.
本資料では,課題内容,理解できなかった部分,作成できなかった機能,自主的に作成した機能について記述する.

\section{課題内容}\label{sec:content}
課題内容はSlackBotプログラムを作成することである.
具体的には以下の2つを行う.

\begin{enumerate}
\item\label{enum:make} 任意の文字列を返信するプログラムの作成

  Slackでユーザが ``「○○○」と言って''と投稿したとき,SlackBotは ``○○○''と返信するプログラムを作成する.
  
\item\label{enum:add} SlackBotプログラムへの機能追加

  Slack以外のWebサービスのAPIやWebhookを利用した機能を追加する.
\end{enumerate}
(\ref{enum:make})(\ref{enum:add})の課題のために作成したプログラムはRubyを利用し,バージョンは2.5.5である.
また,作成したプログラムのコードは208行だった.

\section{理解できなかった部分}
%現時点で理解できなかった部分はな
理解できなかった部分を以下に記述する.
\begin{enumerate}
\item 本プログラムをHerokuにデプロイする際,本プログラム,%Rubyのライブラリを指定するため,
  \verb|Gemfile|,および\verb|Gemfile.lock|が必要だった.
  ここで,\verb|Gemfile|および\verb|Gemfile.lock|は,Rubyプログラムを実行するために必要なgemを指定するファイルである.
  このため,Herokuへのデプロイの際,Gemを指定するファイルは,\verb|Gemfile|もしくは\verb|Gemfile.lock|のいずれか1つでも問題ないと考えた.
  しかし,この場合だと\verb|Gemfile|もしくは\verb|Gemfile.lock|が足りないというエラーが発生する.
  デプロイする際に,なぜ\verb|Gemfile|および\verb|Gemfile.lock|のどちらも必要なのかがわからなかった.
  
%  ここで,\verb|Gemfile.lock|は\verb|Gemfile|があるディレクトリで\verb|bundle install|コマンドを使用することで生成される.

\end{enumerate}


\section{作成できなかった機能}
本課題で作成できなかった機能を以下に記述する.
\begin{enumerate}
\item 本プログラムが,SlackのOutgoing WebHook以外からPOSTリクエストを拒否する機能

%\item 自主的に作成した機能において,``今日'', ``明日'', ``明後日''以外の(日にち)をユーザが入力した際,エラーを示す文字列をSlackBotが返信する機能
  
\end{enumerate}

\section{自主的に作成した機能}
\ref{sec:content}章(\ref{enum:add})の課題のために,自主的に作成した機能を以下に記述する.
\begin{enumerate}
  \item 天気予報の情報を返信する機能

    この機能は,天気予報の情報を返信する機能である.
    ユーザが, ``@masabot (日にち)の(都道府県名)の天気''とSlack上に投稿したとき,SlackBotは指定した日にちと都道府県の天気予報を返信する.
    機能の詳細はSlackBotの仕様書に記載する.
\end{enumerate}


\bibliographystyle{ipsjunsrt}
\bibliography{mybibdata}

\end{document}
